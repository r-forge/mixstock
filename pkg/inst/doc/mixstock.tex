% \VignetteIndexEntry{basic examples of mixed stock analysis}
\documentclass[11pt]{article}
\usepackage{palatino}
\usepackage[american]{babel}
\newcommand{\R}{{\sf R}}
\newcommand{\Splus}{{\sf S-PLUS}}
%\newcommand{\fixme}[1]{\textbf{FIXME: #1}}
\newcommand{\fixme}[1]{}
\usepackage{url}
\usepackage{alltt}
\bibliographystyle{plain}

\newcommand{\Rver}{2.4.1}
\title{Mixed stock analysis in R: getting started with the {\tt mixstock}
package}
\author{Ben Bolker}
\date{\today}
\usepackage{/usr/share/R/share/texmf/Sweave}
\begin{document}
\maketitle

\section{Introduction}
The {\tt mixstock} package is a set of routines
written in the \R\ language
\cite{Rmanual} for doing mixed stock
analysis using
data on markers gathered
from source populations and from one or more mixed populations.
The package was developed for
analyzing mitochondrial DNA (mtDNA) markers from
sea turtle populations, but should be applicable
to any case with discrete sources, discrete mixed
populations, and discrete markers.
(However, I do refer to sources as ``rookeries''
and markers as ``haplotypes'' throughout this document.)
The package is intended to be self-contained,
but some familiarity with \R\ or \Splus\ will
be helpful.  (Some familiarity with your
computer's operating system, which is probably
Microsoft Windows, is also assumed.)
The statistical methods implemented in the
package are described in \cite{bolk+03c} and
\cite{PellMasu01}.

\textbf{This package is in the public domain (GNU General Public
License), is \textcopyright 2007 Ben Bolker and
Toshinori Okuyama,
and comes with NO WARRANTY.  Please suggest improvements to
me (Ben Bolker) at {\tt bolker@zoo.ufl.edu}.}

If you are feeling impatient and confident, turn to
``Quick Start'' (section~\ref{sec:quickstart}).

\section{Installation}
You can skip this section if you are reading this
file via the {\tt vignette()} command in \R --- that
means you've already successfully installed the
package.

To get started, you will have to download and
install the \R\ package, a general-purpose
statistics and graphics package, from
\url{http://cran.us.r-project.org/bin/windows/base/}
if you are in the US
(or see \url{http://www.r-project.org/mirrors.html} for a list of alternative
``mirror sites'' closer to you).
You will download a file called {\tt R-x.y.z-win32.exe}
which will install \R\ for you, when executed; {\tt x.y.z}
stands for the current version of \R\ \Rver\ as of \today).

The following installation instructions assume
you are using a ``modern'' Microsoft Windows system
(tested on 2000 and XP); it is possible to use
\R, and the {\tt mixstock} package, on other
operating systems --- please contact the
authors for more information. 
(The package has been developed under Linux and runs under
Windows; most of it should run under MacOS as well, but
it is not as well supported and you will have to build
the package from sources.   To run hierarchical models using
WinBUGS, you need to have WINE set up on Linux; I'm not
sure about MacOS.)
The setup file is about 17M, and 
\R\ takes up about 40M of disk space.
If you are running an antivirus package that is configured to check
the signatures of executable files before they run, make sure you turn
it off or register the new files installed by \R\ before proceeding.
You may also have some difficulty downloading packages if you have
a firewall running on your computer --- if you have trouble,
you may want to (temporarily, at your own risk!) disable it.

Once you have downloaded and installed \R,
start the \R\ program.  The setup program
should have asked whether you want to add a shortcut
to the desktop or the Start menu: if
you didn't, you will have to search for a file
called {\tt Rgui.exe}, which probably lives somewhere
like {\tt Program Files\\R\\R-\Rver\\bin} depending
on what version of \R\ you are using and where
you decided to install it.
\R\ will open up a window for you with a command
prompt ({\verb+>+}), at which you can type
\R\ commands.  (Don't panic.)

You can exit \R\ by selecting {\tt File/Exit} from
the menus, or by typing {\tt q()} at the command prompt.
In general, if you want help on a particular command
(e.g. {\tt uml}) you can type a question mark
followed by the command name (e.g. {\tt ?uml})

You will next need to install the {\tt mixstock}
package and two other auxiliary packages, over
the WWW, from within \R\ (you will need to
maintain a connection to the internet for this
piece, although it is also possible to do
this step off-line).
Within \R, at the command prompt, type the
following commands:
\begin{Schunk}
\begin{Sinput}
> install.packages("mixstock")
> install.packages("plotrix")
> install.packages("coda")
> install.packages("abind")
> install.packages("R2WinBUGS")
\end{Sinput}
\end{Schunk}

In each case, answer {\tt y} to whether you want to delete
the source files; you won't need them again.
The first command specifies the location of the
{\tt mixstock} package
(the other packages all come from the default source for \R\ packages).
The {\tt install.packages} commands download and
install packages.

(If you don't have a convenient internet connection,
you can also download the .zip files corresponding
to the different packages and install them by going
to the Packages menu within R and choosing
{\tt Install from local zip file}.)

\section{Loading the {\tt mixstock} package and reading  in data}
Start every session with the {\tt mixstock} package by
typing 
\begin{Schunk}
\begin{Sinput}
> library(mixstock)